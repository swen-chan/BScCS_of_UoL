\documentclass{article}
\usepackage{graphicx} % Required for inserting images
\usepackage{amssymb}
\usepackage{mathtools}
\usepackage{amsmath}
\graphicspath{{./images/}}

\makeatletter
\newcommand*{\rom}[1]{\expandafter\@slowromancap\romannumeral #1@}
\makeatother



\title{Midterm of FCS}
\author{Swen Chan}
\date{July 2023}

\begin{document}

\maketitle

\section*{3}
\textbf{(1)}\\
\underline{If the first letter is 'a'}:
The passwords can have 1 lowercase-letter or 2 lowercase-letters.\\
When exists 1 lowercase-letter,number of passwords\(=10*9*8*7*6=30240\);\\
When exists 2 lowercase-letters,number of passwords\(=26*10*9*8*7*C(5,1)=655200\)\\
So when the first letter is 'a',the number of passwords\(=30240+655200=\underline{685440}\)\\\\
\underline{If the first letter is '1'}:
The passwords can have no/1/2 lowercase-letters.\\
When no lowercase-letter,number of passwords\(=9*8*7*6*5=15120\)\\
When exists 1 lowercase-letter,number of passwords\(=26*10*9*8*7*C(5,1)=655200\)\\
When exists 2 lowercase-letters,number of passwords\(=26^2*10*9*8*C(5,2)=4867200\)\\
So when the first letter is '1',the number of passwords\(=15120+655200+4867200=\underline{5537520}\)\\\\
Total number \(=685440+5537520=6222960\)\\
Thus,there are \textbf{6222960} passwords.

\end{document}

