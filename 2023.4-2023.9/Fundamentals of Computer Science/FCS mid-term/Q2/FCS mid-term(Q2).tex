\documentclass{article}
\usepackage{graphicx} % Required for inserting images
\usepackage{amssymb}
\usepackage{mathtools}
\usepackage{amsmath}
\graphicspath{{./images/}}

\makeatletter
\newcommand*{\rom}[1]{\expandafter\@slowromancap\romannumeral #1@}
\makeatother



\title{Midterm of FCS}
\author{Swen Chan}
\date{July 2023}

\begin{document}

\maketitle

\section*{2}
\textbf{Basic step:}\\
when \(n=1,n(n^2+5)=6\),which is devided by 6.\\\\
\textbf{Hypothesis step:}\\
Assuming \(n=k,\forall k\in Z^+,k(k^2+5)\) is divided by 6.\\\\
\textbf{Inductive step:}\\We want to show that \(n=k+1\) such that \((k+1)((k+1)^2+5)\) is also divided by 6,when \(k\in Z^+\).\\
\((k+1)((k+1)^2+5)\\=(k+1)(k^2+2k+6)\\=k(k+1)(k+2)+6(k+1)\)\\
(1)\(6(k+1)\) is divided by 6. \\
(2)\(\frac {k(k+1)(k+2)}{6}=1^2+2^2+...+k^2\in Z^+\) ,\\so \(k(k+1)(k+2)=6(1^2+2^2+...+k^2)\),which is also divided by 6.\\\\
Conclusion: \(\forall n\in Z^+\),6 divides \(n(n^2+5).\)\\\\

\end{document}

